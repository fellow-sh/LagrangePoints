With the deployment of the James Webb Space Telescope, many people are excited to see what discoveries it will uncover about the Universe.
Its development has involved bleeding edge technology and engineering innovations to make its mission objectives possible\autocite{JWSTInnovation}, many of which are not easily understood by the general public. % CITATION NEEDED!
One aspect of the JWST that is not well known is the Lagrange point of which the telescope will be orbiting.
High school physics teaches us that satellites can orbit around stars and planets with gravity.
Yet, when the JWST will orbit around L2, it appears as if the telescope is orbiting around nothing!
How is this possible?
% This introduction needs to be redone.

This investigation aims to locate the collinear Lagrange point, L2, and attempt to model the orbit of a satellite around said point using IB level mathematics as a base and expanding concepts as necessary.
Identifying the locations of the Lagrange points will involve an understanding of algebra and Newtonian mechanics.
The derivation of the halo orbits around the Lagrange points will be more complex, requiring vectors, calculus, and a series of concepts that must be addressed before approaching the solutions to halo orbits.

With that being said, key theorems must be developed from other fields of mathematics that have not yet been covered in IB SL Mathematics.
These theorems will be explored in the preparation section of this paper, as well as a few assertions throughout this investigation.
In this way, this paper will be much more of an adventure to collect the concepts we need before we actually tackle the orbit around L2.