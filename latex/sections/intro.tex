With the deployment of the James Webb Space Telescope, many people are excited to see what discoveries it will uncover about the Universe.
Its development has involved bleeding edge technology and engineering innovations, most of which have the intent to keep the telescope cool. % CITATION NEEDED!
One aspect of the JWST that is not well understood by most people are Lagrange points and how a satellite can be positioned in one of these points.
High school physics teaches us that satellites can orbit around stars and planets because of gravity.
Yet, when the JWST will orbit around Lagrange point 2, it appears as if the telescope is orbiting around nothing!
How is this possible?

This investigation aims to locate the collinear Sun-Earth Lagrange points and attempt to model the orbit of a satellite around a Lagrange point using IB level mathematics.
Identifying the locations of the Lagrange points will involve an understanding of algebra and Newtonian mechanics.
The derivation of the halo orbits around the Lagrange points will be more complex, requiring vectors, calculus, and a bit of linear algebra.