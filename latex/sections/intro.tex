In this day and age, the aerospace industry has become a vital component of the global economy.
In a publication by the Organisation for Economic Co-operation and Development, the space economy plays a key role in the globalization and digitization of the modern world, with space activities making crucial contributions to the social, economic, and scientific aspects of society\autocite{c5996201-en}.
With an increasing investment for spaceflight, there is future need to develop infrastructure that can facilitate the growing space sector, which includes the necessity to understand the nature of orbital mechanics.

Since the longest time, I've always been interested in the development of spaceflight.
There's a sort of cosmic beauty with how objects behave in space, subject predominantly and almost exclusively to the fundamental force that is gravity.
And it is also because of this exclusivity to a single force that makes it ripe for mathematical exploration.
In particular, the launch of the James Webb Space Telescope is a unique mission where the space telescope will sit \textit{behind} the Earth relative to the Sun, in orbit around the Lagrange point L2.
This stands against high school intuition that objects in space orbit around bodies with mass; somehow, the JWST is able to exist in a (somewhat) stable orbit around seemingly nothing!
\textbf{It is for this reason that this investigation aims to explore the mathematics behind Lagrange points.}
More specifically, I seek to learn about the nature of Lagrange points, specifically L2, from a basic physics understanding; an understanding through my mathematical knowledge from school; and from a higher level of mathematics such that we can model what the orbit around a collinear Lagrange point would look like.

It is important to note that this paper does not seek out to give a method which can accurately predict the motion of an object in space with clear solutions; there exists a good reason why orbital mechanics is not taught in detail in high school.
Instead, this paper investigates Lagrange points in such a way that allows for a conceptual understanding that can still facilitate for numerical computations of the motion of objects near a collinear Lagrange point.
It is also due to the complexity of this investigation that more advanced utilities will be used.
Primarily, Python libraries will be used for numerical computations of large numbers, numerical approximation of equations with no solutions, and graphing three-dimensional space.