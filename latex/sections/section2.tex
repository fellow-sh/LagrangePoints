As stated in the introduction, finding out the motion of a satellite around a Lagrange point will be incredibly difficult.
To prepare for this, let us take a step back and tackle these two scenarios.

\subsection{Simple Linear Differential Equations} \label{sec:ode}

For our first scenario, let us imagine that, for some function $n$ with respect to $t$,
\begin{equation}\label{eqn:lde1}
	\frac{dn}{dt} = 4n
\end{equation}
In words, the rate of change of the function is equal to itself multiplied by four. Let us say we wanted to find what function $n$ is.
While it is not as simple to just integrate this equation, we know that $e^x$ is its own derivative.
We also know that, because of the chain rule, $[e^{kt}]' = ke^{kt}$.
If we substitute $n$ for $e^{kx}$ and let $k = 4$, then equation \eqref{eqn:lde1} is true.
Therefore, we can state the following theorem; for some equation
\begin{equation}\label{eqn:theory1}
	\frac{dn}{dt} = kn\ \text{,} \quad n = ce^{kt}
\end{equation}
where $c$ is some initial constant that cannot be expressed from integrating the derivative.
We can verify that this theorem is correct using some algebra and integration.
\begin{align*}
	\frac{dn}{dt} &= kn \\
	\frac{1}{n}\, dn &= k\, dt \\
	\int \frac{1}{n}\, dn &= \int k\, dt \\
	\ln n + C &= kt + D \\
	\ln n &= kt + (D - C) \\
	n &= e^{kt} \cdot e^{D - C} \hspace*{3em}
\end{align*}
This will become handy for equations that are only defined by their derivative.

\subsection{Systems of Linear Differential Equations} \label{sec:ode-sys}

The second scenario builds off the first, and involves equations that are defined by each other.
Consider the following system of equations with functions $x$ and $y$ in terms of $t$:
\begin{align*}
	\frac{dx}{dt} &= 4x - 2y \quad &x(0) = 4 \\
	\frac{dy}{dt} &= 3x - y \quad &y(0) = 2
\end{align*}
Here, the rate of change of $x$ is defined in terms of both $x$ and $y$, and same goes for the rate of change of $y$.
Because these two equations are intertwined, initially it seems impossible to define both $x$ and $y$ as separate from each other.
However, remembering our previous theorem, Equation \eqref{eqn:theory1}, we can predict that both functions will look something like this:
\begin{align}
	x(t) = c_1e^{r_1t} + c_2e^{r_1t} \label{eqn:x-t-general}\\
	y(t) = c_1e^{r_2t} + c_2e^{r_2t} \label{eqn:y-t-general}
\end{align}
Also from Equation \eqref{eqn:theory1}, let us imagine that it is possible to state the rate of change as function of itself multiplied by a constant, $k$.
\begin{align}
	\frac{dx}{dt} &= kx = 4x - 2y \label{eqn:kx-sys}\\
	\frac{dy}{dt} &= ky = 3x - y \label{eqn:ky-sys}
\end{align}
Forgetting the fact that we are involving derivatives for a moment, we can try solving for $k$ by solving this linear system though substitution.
\begin{align}
	4x - 2y &= kx \nonumber \\
	(4 - k)x &= 2y \nonumber \\
	\frac{(4 - k)x}{2} &= y \nonumber \\
	3x - y &=ky \nonumber \\
	3x &= (k + 1)y \nonumber \\
	3x &= (k + 1)\frac{(4 - k)x}{2} \nonumber \\
	3\cdot2 &= (k + 1)(4 - k) \nonumber \\
	0 &= (1 + k)(4 - k) - 3\cdot2 \label{eqn:chr-polynomial-ex}
\end{align}
By now, we can deduce that there will be two possible values for $k$.
Continuing the solution,
\begin{align*}
	0 &= -k^2 - 3k + 2 \\
	0 &= (k - 1)(k - 2)
\end{align*}
Here, the solutions are $k_1 = 1$ and $k_2 = 2$.
We don't know exactly what these values \textit{mean}, other than they correspond to $x$ and $y$, respectively.
Let us see what happens when we plug them into our linear system.
For $k_1$:
\begin{equation*}
	\left. \begin{array}{l}
		(1)x = 4x - 2y \\
		(1)y = 3x - y
	\end{array} \right \}
	\implies 0 = 3x - 2y \implies 2y = 3x
\end{equation*}
This gives us a relationship of $x$ and $y$ where, when $k = 1$, there is a 
$2 : 3$ ratio between $x$ and $y$.
This is significant because, for some specific values of $x$, $y$, and $k$, our system of linear equations \textit{converge} in a manner that essentially defines them as the same equation.
This is easier to understand when graphed.
\begin{figure}[h!]
	\centering
	\captionsetup[subfigure]{justification=centering}
	\begin{subfigure}[b]{0.4\textwidth}
		\centering
		\begin{tikzpicture}
			\draw (-3,0) -- (3,0) node[right] {$x$};
			\draw (0,-3) -- (0,3) node[above] {$y$};
			\draw[thick,green] (-1.5, -3) -- (1.5,3);
			\draw[thick,dashed,red] (-1,-3) -- (1,3);
			\node at (1,-1) [right,align=right] {$\textcolor{green}{4x - 2x = kx}$ \\
				$\textcolor{red}{3x - y = ky}$ \\
				$\textcolor{black}{k = 0}$};
		\end{tikzpicture}
		\caption{\small Linear system with $k = 0$.}
		\label{fig:sample-eigens0}
	\end{subfigure}
	\qquad
	\begin{subfigure}[b]{0.4\textwidth}
		\centering
		\begin{tikzpicture}
			\draw (-3,0) -- (3,0) node[right] {$x$};
			\draw (0,-3) -- (0,3) node[above] {$y$};
			\draw[thick,green] (-2, -3) -- (2,3);
			\draw[thick,dashed,red] (-2,-3) -- (2,3);
			\filldraw[black] (1,1.5) circle (1.25pt) node[anchor=west] {$\textcolor{black}{(2,3)}$};
			\node at (1,-1) [right,align=right] {$\textcolor{green}{4x - 2x = kx}$ \\
				$\textcolor{red}{3x - y = ky}$ \\
				$\textcolor{black}{k = 1}$};
		\end{tikzpicture}
		\caption{\small Linear system with $k = 1$.}
		\label{fig:sample-eigens1}
	\end{subfigure}
	\caption{The linear system defined with regard to $k$. (It is better to think of the graphs as ratios between $x$ and $y$ rather than $y$ as a function of $x$.)}
	\label{fig:sample-eigens}
\end{figure}
In Figure \ref{fig:sample-eigens0}, you can see how, for other values of $k$, the ratios between $x$ and $y$ are different, whereas in Figure \ref{fig:sample-eigens1}, the ratios are the same at $2 : 3$ when $k$ specifically equals 1.
This means we can express the $x$ term of both functions $x$ and $y$ separately for specific values of $k$, which would be 1 and 2.
In terms of Equations \eqref{eqn:x-t-general} and \eqref{eqn:y-t-general}, the $x$ term of the function $x$ will have the additional coefficient of 2 while the $x$ term in the function $y$ will have the additional coefficient of 3.
Therefore, we can add the following to Equations \eqref{eqn:x-t-general} and \eqref{eqn:y-t-general}:
\begin{align}
	x(t) = (2)c_1e^{r_1t} + n_3c_2e^{r_2t} \label{eqn:x-t-partial} \\
	y(t) = (3)c_1e^{r_1t} + n_4c_2e^{r_2t} \label{eqn:y-t-partial}
\end{align}
where $n$ expresses the necessary coefficients for each ratio. What role does $k$ play in the solution? To figure out, first we will find the ratio for when $k = 2$.
\begin{equation*}
	\left. \begin{array}{l}
		(2)x = 4x - 2y \\
		(2)y = 3x - y
	\end{array} \right \}
	\implies 0 = x - y \implies y = x
\end{equation*}
Here, there is a $1 : 1$ ratio between $x$ and $y$. What happens when we enter initial values for $x$ and $y$ that adhere to this ratio? For example, when $x = 2$ and $y = 2$:
\begin{gather*}
	4(2) - 2(2) = 8 - 4 = 4 \\
	3(2) - (2) = 6 - 2 = 4 
\end{gather*} % Do we need to set the equations equal to x and y?
It is as if the initial values were multiplied by the $k$ value!
This is actually expected, given how we laid out Equations \eqref{eqn:kx-sys} and \eqref{eqn:ky-sys}.
In this case, $k$ is the common coefficient between both $x$ and $y$ for the derivative of the function $y$.
\begin{equation*}
	\frac{dy}{dt} = 2y % THIS NEEDS TO BE CORRECTED!!!!
\end{equation*}
This essentially means $k$ can be used as the $r$ values from Equations \eqref{eqn:x-t-general} and \eqref{eqn:y-t-general}.
Finding the functions $x$ and $y$ is almost complete:
\begin{align*}
	x(t) = (2)c_1e^{(1)t} + (1)c_2e^{(2)t} \\
	y(t) = (3)c_1e^{(1)t} + (1)c_2e^{(2)t}
\end{align*}
All that is left is to find the initial constants for each equation given the initial values at the beginning of this scenario.
\begin{align*}
	\left. \begin{array}{l}
		x(0) = 4 = 2c_1 + c_2 \\
		y(0) = 2 = 3c_1 + c_2
	\end{array} \right\}
	&\implies 2 = -c_1 \implies -2 = c_1 \\
	2 = 3(-2) + c_2 &\implies 8 = c_2 \\
	x(t) =& -4e^t + 8e^{2t} \\
	y(t) =& -6e^t + 8e^{2t} 
\end{align*}
And that is the solution!
The reason why we want to understand how to solve this system is because we will encounter a similar system when deriving the equations of motion.
For that reason, we will also need a general method to solving these systems.

\subsection{General Theorems}

The key concepts that have been explored so far are the solution to a simple differential equation and solving a system of linear differential equations using eigenvalues and eigenvectors.
The objective of providing these scenarios is to develop an understanding of the concepts that will be used.
And indeed, the equations of motion that govern the movement around a Lagrange point are differential equations and will require the concepts above to solve numerically.
However, generalizing the concepts that have been applied so far, so that they can be applied to the equations of motion, will involve more concepts that will over-complicate this paper.
Therefore, the general theorems that will be applied for the equations of motion must sometimes be asserted.

Before the general theorems are established, is important to understand is how these concepts were used.
The first concept to know about is a differential equation; when an unknown function is defined alongside its derivative.
An applicable example of one is the simple equation explored in subsection \ref{sec:ode}, where the solution is an exponential function and whose solution is generalized in Equation \eqref{eqn:theory1}.
The second concept is a system of differential equations, where not only are there functions defined with relation to their derivatives, but also with relation to each other.
Equations \eqref{eqn:kx-sys} and \eqref{eqn:ky-sys} are a prime example of this.
The approach to the solution of these kinds of systems involves the third important concept, which is eigenvalues and eigenvectors.
The values of $k$ that were found in section \ref{sec:ode-sys} are known as eigenvalues and the ratios that were found alongside them are called eigenvectors.
The eigenvalues express, as single coefficients, how the system of equations effectively changes given values for each function with respect to the eigenvectors; the ratios between each function.
We can infer from Equation \eqref{eqn:chr-polynomial-ex} that, for two linear equations,
\begin{align*}
	\lambda x_1 &= ax_1 + bx_2 \\
	\lambda x_2 &= cx_1 + dx_2
\end{align*}
The eigenvalues, $\lambda$, are calculated as:
\begin{equation}
	0 = (a - \lambda)(d - \lambda) - bc
\end{equation}
For a system of three variables and three differential equations:
\begin{align*}
	\lambda x_1 &= ax_1 + bx_2 + cx_3 \\
	\lambda x_2 &= dx_1 + ex_2 + fx_3 \\
	\lambda x_3 &= gx_1 + hx_2 + ix_3
\end{align*}
The eigenvalues are calculated as:
\begin{align*}
	0 &= (a - \lambda)[(e - \lambda)(i - \lambda) - fh] + b[d(i - \lambda) - fg] - c[dh - (e - \lambda)g] \\
	&= (a - \lambda)(e - \lambda)(i - \lambda) + bfg + cdh - cg(e - \lambda) - bd(i - \lambda) - fh(a - \lambda)
\end{align*}
Notice how the calculation is, in essence, the calculation for the eigenvalues of three two-equation systems multiplied by the coefficients of the first top-most equation, $a, b, \text{and}\, c$.
This behaviour can be extrapolated onto the solution for a system of four variables and four differential equations with coefficients $(a_1, a_2, \dots, a_n)$,
%The key concepts explored through these scenarios are the solutions to systems of linear differential equations, and eigenvalues and eigenvectors.
%The first concept, differential equations, will appear when solving for the equations of motion and it is imperative that we understand how to approach them.
%We already know how to approach a single linear differential equation as stated from Equation \eqref{eqn:theory1}. However, when there are multiple differential equations that are defined in terms of each other, we will need eigenvalues and eigenvectors.

%As for the second concept, eigenvalues are the values of $k$ that we found, while eigenvectors are those special ratios between $x$ and $y$ where our system of equations converged to a single line on our graph back in Figure \ref{fig:sample-eigens1}.
%They are frequently discussed in linear algebra, where eigenvectors are specific vectors that maintain their direction after a linear transformation, and eigenvalues are the factor which the linear transformation scales the eigenvectors.
%However, linear algebra is significantly beyond the concern of this investigation.
%Eigenvalues and eigenvectors will be important for solving multiple differential equations who are defined by each other.
%We can infer from Equation \eqref{eqn:chr-polynomial-ex} that, for two linear differential equations,
%\begin{align*}
%	\frac{dx_1}{dt} = ax_1 + bx_2 \\
%	\frac{dx_2}{dt} = cx_1 + dx_2
%\end{align*}
%The eigenvalues, $\lambda$, are calculated as:
%\begin{equation}
%	0 = (a - \lambda)(d - \lambda) - bc
%\end{equation}
%We will need to generalize this for any number of differential equations, given that there exists only one differential equation for each unknown function.
%At this point, we have explored the concepts necessary to understand the solutions to systems of differential equations, and further exploration will expand beyond the aims of this paper.