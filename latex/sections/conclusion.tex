Through this investigation, we have used three levels of mathematical understanding; through kinematic and algebraic means; through calculus, vectors, and mechanics; and through numerical computation and approximation.
In doing so, we have understood that Lagrange points are where the gravitational and rotational forces create regions with a degree of equilibrium.
And, through finding the equations of motion, we were able to get an idea of how an object behaves around Lagrange points.

The strategy towards investigating the mathematics of Lagrange points involved approaching the concept from increasing degree of complexity.
This method of approaching a complicated subject through simpler parts, than only adjust for specific values, is a useful problem solving technique that I hope I can use to understand other concepts in mathematics.
In addition, through the research that was involved in the making of this paper, I have been introduce to a large range of concepts that have yet to be learned, such as linear algebra and eigenvalues, diagonalization, differential equations, numerical approximation of differential equations, linearization, and stability analysis.
In particular, differential equations are applicable to a multitude of scenarios in the world, such as the growth rate of populations, effects of resistance in motion, chaos theory, and seemingly infinite number of other real world instances.
And while I may not understand them now, I look forward to learning in post-secondary education.

However, it is also because of the limitations of my own mathematical knowledge that this investigation has its flaws.
The equations of motion, in particular, could have its own investigation to its properties and relevance to other problems in physics, notably centrifugal and Coriolis forces.
Additionally, the equations of motions do have solutions through linear analysis, but such an analysis would require a level of mathematics that is well beyond what is taught in IB.
Also, because of numerical approximation of differential equations is a field of its own in mathematics, I have had to omit a number of details on how the computation works and where it comes from.
It is remarkable how not only the concept of an orbit can become increasing complex, but how it also requires increasing levels of abstraction to simply comprehend it completely.
There are many other things which I wanted to explore in this investigation but could not include due for the sake of brevity.
Regardless, this investigation has helped me understand a great deal about Lagrange points and their significance for future spaceflight, and I hope to apply this understanding to create ideas and solutions in other areas of knowledge.
